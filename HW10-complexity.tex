
%%%%%%%%%%%%%%%%%%%%%%%%%%%%%%%%%%%%%%%%%%%%%%%%%%%%%%%%%%
%%
%% This is the "PREAMBLE". Here we define the type of document and load in any packages we might want. You can also set parameters %% % and create your own short-hand here.
%%
%%%%%%%%%%%%%%%%%%%%%%%%%%%%%%%%%%%%%%%%%%%%%%%%%%%%%%%%%%

 \documentclass[9pt]{article}
 
 \def\solutions{1}


 \usepackage{amsmath}
 \usepackage{amssymb}
 \usepackage{graphicx}    % needed for including graphics e.g. EPS, PS \usepackage{tikz}
 \usepackage{tikz}
 \usepackage{tikzsymbols}
 \usepackage{relsize}
 \usetikzlibrary{patterns,decorations.pathreplacing,shapes,arrows}
 \usepackage{algorithm}     % creates the float "Algorithm 1: My Algorithm"
 \usepackage[noend]{algorithmic}   % for the algorithm itself, e.g. "for x = 1 to n..."
 % \algtext*{EndWhile}% Remove "end while" text
 % \algtext*{EndIf}% Remove "end if" text \topmargin -2.5cm        % read Lamport p.163
 \oddsidemargin -0.04cm   % read Lamport p.163
 \evensidemargin -0.04cm  % same as oddsidemargin but for left-hand pages
 \textwidth 16.59cm
 \textheight 25.94cm
% \pagestyle{empty}        % Uncomment if don't want page numbers
 \pagenumbering{gobble}
 \parskip 7.2pt           % sets spacing between paragraphs
 %\renewcommand{\baselinestretch}{1.5} 	% Uncomment for 1.5 spacing between lines
 \parindent 0pt		  % sets leading space for paragraphs

% No date in header
\date{}

\usepackage{hyperref}
\hypersetup{
    colorlinks=true,
    linkcolor=blue,
    filecolor=magenta,      
    urlcolor=cyan,
}
\usepackage{amsthm}
\usepackage{fancyhdr}
\pagestyle{fancy}
\setlength{\headsep}{36pt}

\usepackage{hyperref}

\newcommand{\lp}{\left(}
\newcommand{\rp}{\right)}
\newcommand{\lb}{\left[}
\newcommand{\rb}{\right]}
\newcommand{\ls}{\left\{}
\newcommand{\rs}{\right\}}
\newcommand{\lbar}{\left|}
\newcommand{\rbar}{\right|}
\newcommand{\ld}{\left.}
\newcommand{\rd}{\right.}

\newcommand{\myexists}{\exists \hspace{.3mm}}

\newcommand{\hs}{\hspace{.75mm}}
\newcommand{\bs}{\hspace{-.75mm}}
\newcommand{\nin}{\noindent}

\newcommand{\fx}{f\bs\left( x \right)}
\newcommand{\gx}{g\bs\left( x \right)}
\newcommand{\qx}{q\bs\left( x \right)}

\newcommand{\nn}{\nonumber}

\newcommand{\vfive}{\vspace{5mm}}
\newcommand{\vthree}{\vspace{3mm}}

\newcommand{\fof}[1]{f\lp #1\rp}
\newcommand{\gof}[1]{g\lp #1\rp}
\newcommand{\qof}[1]{q\lp #1\rp}

\newcommand{\myp}[1]{\left( #1 \right)}
\newcommand{\myb}[1]{\left[ #1 \right]}
\newcommand{\mys}[1]{\left\{ #1 \right\}}
\newcommand{\myab}[1]{\left| #1 \right|}

\newcommand{\myj}{_j}
\newcommand{\myjp}{_{j+1}}
\newcommand{\myjm}{_{j-1}}

\newcommand{\f}[1]{f\hspace{-1mm}\left( #1 \right)}
\newcommand{\fp}[1]{f'\hspace{-1mm}\left( #1 \right)}
\newcommand{\g}[1]{g\hspace{-1mm}\left( #1 \right)}
\newcommand{\gp}[1]{g'\hspace{-1mm}\left( #1 \right)}
\newcommand{\q}[1]{q\hspace{-1mm}\left( #1 \right)}
\newcommand{\qp}[1]{q'\hspace{-1mm}\left( #1 \right)}
\newcommand{\Px}[1]{P\hspace{-1mm}\left( x_{#1} \right)}
\newcommand{\Qx}[1]{Q\hspace{-1mm}\left( x_{#1} \right)}

\newcommand{\tten}[1]{\times 10^{#1}}

\newcommand{\aij}[1]{a_{#1}}
\newcommand{\bij}[1]{b_{#1}}
\newcommand{\rij}[1]{r_{#1}}

\newcommand{\R}[1]{\mathbb{R}^{#1}}

\newcommand{\ith}{i^{\textrm{th}}}
\newcommand{\jth}{i^{\textrm{th}}}
\newcommand{\kth}{i^{\textrm{th}}}

\newcommand{\inv}[1]{{#1}^{-1}}

\newcommand{\bx}{\mathbf{x}}
\newcommand{\bv}{\mathbf{v}}
\newcommand{\bw}{\mathbf{w}}
\newcommand{\by}{\mathbf{y}}
\newcommand{\bb}{\mathbf{b}}
\newcommand{\be}{\mathbf{e}}
\newcommand{\br}{\mathbf{r}}
\newcommand{\xhat}{\hat{\mathbf{x}}}

\newcommand{\beq}{\begin{eqnarray}}
\newcommand{\eeq}{\end{eqnarray}}

\newcommand{\ben}{\begin{enumerate}}
\newcommand{\een}{\end{enumerate}}

\newcommand{\bsq}{\mathsmaller{\blacksquare}}

\newcommand{\iter}[1]{^{\myp{#1}}}

% matrix macro
\newcommand{\mymat}[1]{
\left[
\begin{array}{rrrrrrrrrrrrrrrrrrrrrrrrrrrrrrrrrrrrrrr}
#1
\end{array}
\right]
}

\newcommand{\makenonemptybox}[2]{%
%\par\nobreak\vspace{\ht\strutbox}\noindent
\item[]
\fbox{% added -2\fboxrule to specified width to avoid overfull hboxes
% and removed the -2\fboxsep from height specification (image not updated)
% because in MWE 2cm is should be height of contents excluding sep and frame
\parbox[c][#1][t]{\dimexpr\linewidth-2\fboxsep-2\fboxrule}{
  \hrule width \hsize height 0pt
  #2
 }%
}%
\par\vspace{\ht\strutbox}
}
\makeatother

\newcommand{\smallaug}[1]{
\left[
\begin{array}{rr|r}
#1
\end{array}
\right]
}

%%%%%%%%%%%%%%%%%%%%%%%%%%%%%%%%%%%%%%%%%%%%%%%%%%%%%%%%%%
%%
%% End of PREAMBLE
%%
%%%%%%%%%%%%%%%%%%%%%%%%%%%%%%%%%%%%%%%%%%%%%%%%%%%%%%%%%%



% ======================================================================================
% Actual document starts here. 
% PLEASE FILL IN YOUR NAME AND STUDENT ID.
% ======================================================================================
\begin{document}

\lhead{{\bf CSCI 3104, Algorithms \\ Problem Set 10 (50 points)} }
\rhead{Name: \fbox{YOUR NAME HERE} \\ ID: \fbox{YOUR STUDENT ID HERE} \\ {\bf Due THURSDAY, APRIL 29, 2021 \\ Spring 2021, CU-Boulder}}
\renewcommand{\headrulewidth}{0.5pt}

\phantom{Test}

\begin{small}
\textit{Advice 1}:\ For every problem in this class, you must justify your answer:\ show how you arrived at it and why it is correct. If there are assumptions you need to make along the way, state those clearly.
\vspace{-3mm} 

\textit{Advice 2}:\ Verbal reasoning is typically insufficient for full credit. Instead, write a logical argument, in the style of a mathematical proof.\\
\vspace{-3mm} 

\textbf{Instructions for submitting your solution}:
\vspace{-5mm} 

\begin{itemize}
	\item The solutions \textbf{should be typed} and we cannot accept hand-written solutions. \href{http://ece.uprm.edu/~caceros/latex/introduction.pdf}{Here's a short intro to Latex.}
	\item You should submit your work through \href{https://www.gradescope.com/courses/218966}{\textbf{Gradescope}} only.
	\item The easiest way to access Gradescope is through our Canvas page. There is a Gradescope button in the left menu.
	\item Gradescope will only accept \textbf{.pdf} files.
	\item \href{https://www.youtube.com/watch?v=u-pK4GzpId0&feature=emb_logo}{It is vital that you match each problem part with your work.} Skip to 1:40 to just see the matching info.
\end{itemize}
\vspace{-4mm} 
\end{small}

\hrulefill
\pagebreak



\ben
%%%%%%%%%%%%%%%%%%%%%%%%%%%%%%%%%%%%%%%%%%%%%%%%%%%%%%%%
% PROBLEM  ONE %% PROBLEM  ONE %% PROBLEM  ONE %% PROBLEM  ONE %% PROBLEM  ONE %
%==============================================================================
% Problem 1:Topic 1
%==============================================================================
% PROBLEM  ONE %% PROBLEM  ONE %% PROBLEM  ONE %% PROBLEM  ONE %% PROBLEM  ONE %
%%%%%%%%%%%%%%%%%%%%%%%%%%%%%%%%%%%%%%%%%%%%%%%%%%%%%%%%

\item Let $P_1, P_2$ be two problems such that $P_1 \leq_p P_2$. That is, we have a polynomial-time reduction \\$r : P_1 \rightarrow P_2$. If we assume $P_2 \in P$, explain why this implies that $P_1 \in P$.

\item Recall the $k$-\textsf{Colorability} problem.
\begin{itemize}
\item \textsf{Input:} Let $G$ be a simple, undirected graph.
\item \textsf{Decision:} Can we color the vertices of $G$ using exactly $k$ colors, such that whenever $u$ and $v$ are adjacent vertices, $u$ and $v$ receive different colors?
\end{itemize}

\noindent It is well known that $k$-\textsf{Colorability} belongs to NP, and $3$-\textsf{Colorability} is NP-complete. Our goal is to show that the $4$-\textsf{Colorability} problem is also NP-complete. To do so, we reduce the $3$-\textsf{Colorability} problem to the $4$-\textsf{Colorability} problem. We provide the reduction below. The questions following the reduction will ask you to verify its correctness.

\textbf{Reduction:} Let $G$ be a simple, undirected graph. We construct a new simple, undirected graph $H$ by starting with a copy of $G$. We then add a new vertex $t$ to $H$, and for each vertex $v \in V(G)$ we add the edge $tv$ to $E(H)$.

\noindent \textbf{Your job} is to verify the correctness of the reduction by completing parts (a)--(c) below, which completes the proof that $4$-\textsf{Colorability} is NP-complete. For the rest of this problem, let $G$ be a graph, and let $H$ be the result of applying the reduction to $G$.


\ben
  \item Suppose that $G$ is colorable using exactly $3$ colors. Argue that $H$ is colorable using exactly $4$ colors.
  \item Suppose that $H$ is colorable using exactly $4$ colors. Argue that $G$ is colorable using exactly $3$ colors.
  \item Let $n$ be the number of vertices in $G$. Carefully explain why $H$ can be constructed in time polynomial in $n$. [\textbf{Hint:} Count the number of vertices and edges we add to $G$ in order to obtain $H$.]
\een

\een 


\end{document}
