
%%%%%%%%%%%%%%%%%%%%%%%%%%%%%%%%%%%%%%%%%%%%%%%%%%%%%%%%%%
%%
%% This is the "PREAMBLE". Here we define the type of document and load in any packages we might want. You can also set parameters %% % and create your own short-hand here.
%%
%%%%%%%%%%%%%%%%%%%%%%%%%%%%%%%%%%%%%%%%%%%%%%%%%%%%%%%%%%

 \documentclass[9pt]{article}
 
 \def\solutions{1}
\usepackage{mathtools}
\DeclarePairedDelimiter\ceil{\lceil}{\rceil}
\DeclarePairedDelimiter\floor{\lfloor}{\rfloor}

\usepackage{hyperref} 
 \usepackage{amsmath}
 \usepackage{amssymb}
 \usepackage{graphicx}    % needed for including graphics e.g. EPS, PS \usepackage{tikz}
 \usepackage{tikz}
 \usepackage{tikzsymbols}
 \usepackage{relsize}
 \usetikzlibrary{patterns,decorations.pathreplacing,shapes,arrows}
 \usepackage{algorithm2e}
 \topmargin -2.5cm        % read Lamport p.163
 \oddsidemargin -0.04cm   % read Lamport p.163
 \evensidemargin -0.04cm  % same as oddsidemargin but for left-hand pages
 \textwidth 16.59cm
 \textheight 25.94cm
% \pagestyle{empty}        % Uncomment if don't want page numbers
 \pagenumbering{gobble}
 \parskip 7.2pt           % sets spacing between paragraphs
 %\renewcommand{\baselinestretch}{1.5} 	% Uncomment for 1.5 spacing between lines
 \parindent 0pt		  % sets leading space for paragraphs

% No date in header
\date{}

\usepackage{hyperref}
\hypersetup{
    colorlinks=true,
    linkcolor=blue,
    filecolor=magenta,      
    urlcolor=cyan,
}
\usepackage{amsthm}
\usepackage{fancyhdr}
\pagestyle{fancy}
\setlength{\headsep}{36pt}

\usepackage{hyperref}

\newcommand{\lp}{\left(}
\newcommand{\rp}{\right)}
\newcommand{\lb}{\left[}
\newcommand{\rb}{\right]}
\newcommand{\ls}{\left\{}
\newcommand{\rs}{\right\}}
\newcommand{\lbar}{\left|}
\newcommand{\rbar}{\right|}
\newcommand{\ld}{\left.}
\newcommand{\rd}{\right.}

\newcommand{\myexists}{\exists \hspace{.3mm}}

\newcommand{\hs}{\hspace{.75mm}}
\newcommand{\bs}{\hspace{-.75mm}}
\newcommand{\nin}{\noindent}

\newcommand{\fx}{f\bs\left( x \right)}
\newcommand{\gx}{g\bs\left( x \right)}
\newcommand{\qx}{q\bs\left( x \right)}

\newcommand{\nn}{\nonumber}

\newcommand{\vfive}{\vspace{5mm}}
\newcommand{\vthree}{\vspace{3mm}}

\newcommand{\fof}[1]{f\lp #1\rp}
\newcommand{\gof}[1]{g\lp #1\rp}
\newcommand{\qof}[1]{q\lp #1\rp}

\newcommand{\myp}[1]{\left( #1 \right)}
\newcommand{\myb}[1]{\left[ #1 \right]}
\newcommand{\mys}[1]{\left\{ #1 \right\}}
\newcommand{\myab}[1]{\left| #1 \right|}

\newcommand{\myj}{_j}
\newcommand{\myjp}{_{j+1}}
\newcommand{\myjm}{_{j-1}}

\newcommand{\f}[1]{f\hspace{-1mm}\left( #1 \right)}
\newcommand{\fp}[1]{f'\hspace{-1mm}\left( #1 \right)}
\newcommand{\g}[1]{g\hspace{-1mm}\left( #1 \right)}
\newcommand{\gp}[1]{g'\hspace{-1mm}\left( #1 \right)}
\newcommand{\q}[1]{q\hspace{-1mm}\left( #1 \right)}
\newcommand{\qp}[1]{q'\hspace{-1mm}\left( #1 \right)}
\newcommand{\Px}[1]{P\hspace{-1mm}\left( x_{#1} \right)}
\newcommand{\Qx}[1]{Q\hspace{-1mm}\left( x_{#1} \right)}

\newcommand{\tten}[1]{\times 10^{#1}}

\newcommand{\aij}[1]{a_{#1}}
\newcommand{\bij}[1]{b_{#1}}
\newcommand{\rij}[1]{r_{#1}}

\newcommand{\R}[1]{\mathbb{R}^{#1}}

\newcommand{\ith}{i^{\textrm{th}}}
\newcommand{\jth}{i^{\textrm{th}}}
\newcommand{\kth}{i^{\textrm{th}}}

\newcommand{\inv}[1]{{#1}^{-1}}

\newcommand{\bx}{\mathbf{x}}
\newcommand{\bv}{\mathbf{v}}
\newcommand{\bw}{\mathbf{w}}
\newcommand{\by}{\mathbf{y}}
\newcommand{\bb}{\mathbf{b}}
\newcommand{\be}{\mathbf{e}}
\newcommand{\br}{\mathbf{r}}
\newcommand{\xhat}{\hat{\mathbf{x}}}

\newcommand{\beq}{\begin{eqnarray}}
\newcommand{\eeq}{\end{eqnarray}}

\newcommand{\ben}{\begin{enumerate}}
\newcommand{\een}{\end{enumerate}}

\newcommand{\bsq}{\mathsmaller{\blacksquare}}

\newcommand{\iter}[1]{^{\myp{#1}}}

% matrix macro
\newcommand{\mymat}[1]{
\left[
\begin{array}{rrrrrrrrrrrrrrrrrrrrrrrrrrrrrrrrrrrrrrr}
#1
\end{array}
\right]
}

\newcommand{\makenonemptybox}[2]{%
%\par\nobreak\vspace{\ht\strutbox}\noindent
\item[]
\fbox{% added -2\fboxrule to specified width to avoid overfull hboxes
% and removed the -2\fboxsep from height specification (image not updated)
% because in MWE 2cm is should be height of contents excluding sep and frame
\parbox[c][#1][t]{\dimexpr\linewidth-2\fboxsep-2\fboxrule}{
  \hrule width \hsize height 0pt
  #2
 }%
}%
\par\vspace{\ht\strutbox}
}
\makeatother

\newcommand{\smallaug}[1]{
\left[
\begin{array}{rr|r}
#1
\end{array}
\right]
}

%%%%%%%%%%%%%%%%%%%%%%%%%%%%%%%%%%%%%%%%%%%%%%%%%%%%%%%%%%
%%
%% End of PREAMBLE
%%
%%%%%%%%%%%%%%%%%%%%%%%%%%%%%%%%%%%%%%%%%%%%%%%%%%%%%%%%%%



% ======================================================================================
% Actual document starts here. 
% PLEASE FILL IN YOUR NAME AND STUDENT ID.
% ======================================================================================
\begin{document}

\lhead{{\bf CSCI 3104, Algorithms \\ Problem Set 05 (50 points)} }
\rhead{Name: \fbox{YOUR NAME HERE} \\ ID: \fbox{YOUR STUDENT ID HERE} \\ {\bf Due Feb 19, 2021 \\ Spring 2021, CU-Boulder}}
\renewcommand{\headrulewidth}{0.5pt}

\phantom{Test}

\begin{small}
\textit{Advice 1}:\ For every problem in this class, you must justify your answer:\ show how you arrived at it and why it is correct. If there are assumptions you need to make along the way, state those clearly.
\vspace{-3mm} 

\textit{Advice 2}:\ Verbal reasoning is typically insufficient for full credit. Instead, write a logical argument, in the style of a mathematical proof.\\
\vspace{-3mm} 

\textbf{Instructions for submitting your solution}:
\vspace{-5mm} 

\begin{itemize}
	\item The solutions \textbf{should be typed} and we cannot accept hand-written solutions. \href{http://ece.uprm.edu/~caceros/latex/introduction.pdf}{Here's a short intro to Latex.}
	\item You should submit your work through \href{https://www.gradescope.com/courses/218966}{\textbf{Gradescope}} only.
	\item The easiest way to access Gradescope is through our Canvas page. There is a Gradescope button in the left menu.
	\item Gradescope will only accept \textbf{.pdf} files.
	\item \href{https://www.youtube.com/watch?v=u-pK4GzpId0&feature=emb_logo}{It is vital that you match each problem part with your work.} Skip to 1:40 to just see the matching info.
\end{itemize}
\vspace{-4mm} 
\end{small}

\hrulefill
\pagebreak



\ben
%%%%%%%%%%%%%%%%%%%%%%%%%%%%%%%%%%%%%%%%%%%%%%%%%%%%%%%%
% PROBLEM  ONE %% PROBLEM  ONE %% PROBLEM  ONE %% PROBLEM  ONE %% PROBLEM  ONE %
%==============================================================================
% Problem 1:Topic 1
%==============================================================================
% PROBLEM  ONE %% PROBLEM  ONE %% PROBLEM  ONE %% PROBLEM  ONE %% PROBLEM  ONE %
%%%%%%%%%%%%%%%%%%%%%%%%%%%%%%%%%%%%%%%%%%%%%%%%%%%%%%%%

\item {\itshape (16 pts) Suppose in quicksort, we have access to an algorithm which chooses a pivot such that, the ratio of the size of the two subarrays divided by the pivot is a \textbf{constant} $k$. i.e an array of size $n$ is divided into two arrays, the first array is of size $n_1 = \frac{nk}{k+1}$ and the second array is of size $n_2 = \frac{n}{k+1}$ so that the ratio $\frac{n_1}{n_2} = k$ a constant. 
	}
	\begin{enumerate}%[label=(\alph)]
	
	\item{\itshape
    (3 pts) Given an array, what value of $k$ will result in the best partitioning?}
    
    \item{\itshape
    (10 pts) Write down a recurrence relation for this version of QuickSort, and solve it asymptotically using \textbf{recursion tree} method to come up with a big-O notation. For this part of the question assume $k= 3$. Show your work, write down the first few levels of the tree, identify the pattern and solve. Assume that the time it takes to find the pivot is $\Theta$(n) for lists of length $n$. Note: Remember that a big-O bound is just an upper bound. So come up with an expression and make arguments based on the big-O notation definition. }
	
	\item{ \itshape
(3 pts) Does the value of $k$ affect the running time?}
		\end{enumerate}
  \if\solutions1
  \vspace{2mm}
  
  \textbf{Solution:}   \\
%==============================================================================
% STUDENTS: TYPE YOUR SOLUTIONS HERE. (Between \textbf{Solution:} and \fi )
%==============================================================================


\fi

\newpage


%%%%%%%%%%%%%%%%%%%%%%%%%%%%%%%%%%%%%%%%%%%%%%%%%%%%%%%%
% PROBLEM TWO %% PROBLEM TWO %% PROBLEM TWO %% PROBLEM TWO %% PROBLEM TWO %
%==============================================================================
% Problem 2: Topic 2
%==============================================================================
% PROBLEM TWO %% PROBLEM TWO %% PROBLEM TWO %% PROBLEM TWO %% PROBLEM TWO %
%%%%%%%%%%%%%%%%%%%%%%%%%%%%%%%%%%%%%%%%%%%%%%%%%%%%%%%%
\vspace{5mm}
\item { \itshape 
	    (10 pts) Consider a chaining hash table $A$ with $b$ slots that holds data from a fixed, finite universe $U$. }
		\begin{enumerate}
		\item {\itshape (3 pts) State the simple uniform hashing assumption. }
		
		\item {\itshape (7 pts) Consider the worst case analysis of hash tables. Suppose we start with an empty hash table, $A$. A \textbf{collision} occurs when an element is hashed into a slot where there is another element already. Assume that $|U|$ represents the size of the universe and $b$ represents the number of slots in the hash table. Let us assume that $|U|\le b$. Suppose we intend to insert $n$ elements into $A$ \textbf{Do not assume the simple uniform hashing assumption for this subproblem.}
		\begin{enumerate}
	      \item What is the worst case for the number of collisions? Express your answer in terms of $n$.
	      \item What is the load factor for $A$ in the previous question?
	      \item How long will a successful search take, on average? Give a big-Theta bound.
	\end{enumerate}}
		
    \end{enumerate}
\textbf{Solution:} \\
%==============================================================================
% STUDENTS: TYPE YOUR SOLUTIONS HERE. (Between \textbf{Solution:} and \fi )
%==============================================================================




\fi
\newpage


%%%%%%%%%%%%%%%%%%%%%%%%%%%%%%%%%%%%%%%%%%%%%%%%%%%%%%%%
% PROBLEM THREE %% PROBLEM THREE %% PROBLEM THREE %% PROBLEM THREE %% PROBLEM THREE %
%==============================================================================
% Problem 3: Topic 3
%==============================================================================
% PROBLEM THREE %% PROBLEM THREE %% PROBLEM THREE %% PROBLEM THREE %% PROBLEM THREE %
%%%%%%%%%%%%%%%%%%%%%%%%%%%%%%%%%%%%%%%%%%%%%%%%%%%%%%%%

\vspace{5mm}

\item { \itshape 
	    (12 pts) Consider a hash table of size 100 with slots from 1 to 100. Consider the hash function $h(k) = \floor*{100k}$ for all keys $k$ for a table of size 100. You have three applications.
	    \begin{itemize}
	        \item \textbf{Application 1}: Keys are  generated uniformly at random from the interval $[0.3, 0.8]$.
	        \item \textbf{Application 2}: Keys are  generated uniformly at random from the interval $[0.1, 0.4] \cup [0.6, 0.9]$.
	        \item \textbf{Application 3}: Keys are  generated uniformly at random from the interval $[0, 1]$.
	    \end{itemize}
	}
		\begin{enumerate}
		\item {\itshape (3 pts) Suppose you have $n$ keys in total chosen for each application. What is the resulting load factor $\alpha$ for each application? }
		
		\item {\itshape (3 pts) Which application will yield the worst performance? }
		
		\item {\itshape (3 pts) Which application will yield the best performance? }
		
		\item {\itshape (3 pts) Which application will allow the uniform hashing property to apply? }
		
    \end{enumerate}
\if\solutions1
\vspace{2mm}




\if\solutions1
\vspace{2mm}




\if\solutions1
\vspace{2mm}

\textbf{Solution:} \\
%==============================================================================
% STUDENTS: TYPE YOUR SOLUTIONS HERE. (Between \textbf{Solution:} and \fi )
%==============================================================================


\fi

\newpage

%%%%%%%%%%%%%%%%%%%%%%%%%%%%%%%%%%%%%%%%%%%%%%%%%%%%%%%%
% PROBLEM FOUR %% PROBLEM FOUR %% PROBLEM FOUR %% PROBLEM FOUR %% PROBLEM FOUR %
%==============================================================================
% Problem 4: Root and Ratio Test Review
%==============================================================================
% PROBLEM FOUR %% PROBLEM FOUR %% PROBLEM FOUR %% PROBLEM FOUR %% PROBLEM FOUR %
%%%%%%%%%%%%%%%%%%%%%%%%%%%%%%%%%%%%%%%%%%%%%%%%%%%%%%%%


\vspace{5mm}

\item{\itshape (12 pts) Median of Medians Algorithm
	}
	\begin{enumerate}%[label=(\alph)]
	
	\item{\itshape
    (4 pts) Illustrate how to apply the QuickSelect algorithm to find the $k = 4$th smallest element in the given array: \texttt{A = [5, 3, 4, 9, 2, 8, 1, 7, 6]} by showing the recursion call tree. Refer to  \href{https://canvas.colorado.edu/courses/69354/files/28668920?wrap=1}{Sam's Lecture 10} for notes on QuickSelect algorithm works }
    
    \item{\itshape
    (4 pt) Explain in 2-3 sentences the purpose of the Median of Medians algorithm.}
	
	\item{ \itshape
(4 pts)Consider applying Median of Medians algorithm (A Deterministic QuickSelect algorithm) to find the $4$th largest element in the following array: \texttt{A = [6, 10, 80, 18, 20, 82, 33, 35, 0, 31, 99, 22, 56, 3, 32, 73, 85, 29, 60, 68, 99, 23, 57, 72, 25]}.Illustrate how the algorithm would work for the first two recursive calls and indicate which sub array would the algorithm continue searching following the second recursion. Refer to \href{https://canvas.colorado.edu/courses/69354/files/28587193?wrap=1}{Rachel's Lecture 8} for notes on Median of Medians Algorithm}
		\end{enumerate}
\if\solutions1
\vspace{3mm}
{\bf Solution}: \\
%==============================================================================
% STUDENTS: TYPE YOUR SOLUTIONS HERE. (Between \textbf{Solution:} and \fi )
%==============================================================================



\fi


%========================================================================================================================

\een 


\end{document}
